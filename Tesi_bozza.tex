\documentclass[a4paper]{article}
\usepackage{graphicx}
\usepackage{caption}
\usepackage{subcaption}
\usepackage{wrapfig}
\usepackage[utf8]{inputenc}
\usepackage[margin=1in]{geometry}
\usepackage{amsmath}




\begin{document}
	\centerline{\sc Bozza Tesi}
	\centerline{\sc \large Routing di sorgente in reti DTN deterministiche}
	\centerline{\sc Luca Olivieri}
	\vspace{2pc}
	
	\clearpage
	
	\section*{Ringraziamenti}	
	
	\clearpage
	
	\section*{Estratto}
	
	\clearpage
	
	\tableofcontents
	
	\clearpage
	
	\section{Introduzione}
	
	\clearpage
	
	\section{Delay Tolerant Networks}
		
		\subsection{Origini e idea generale}
		
		Le reti convenzionali mal si adattano in condizioni ambientali estreme e in recenti anni questo problema è stato oggetto di ricerca. I principali limiti dei dispositivi operanti in queste circostanze riguardano la capacità di rete limitata, la scarsa disponibilità di energia e memoria, il movimento e in particolare lunghi ritardi e connettività intermittente. Queste ultime limitazioni saranno la caratteristica principale di questo lavoro, incentrato su reti di tipo satellitare dove il costante movimento orbitale impone una connettività intermittente e la notevole distanza tra i nodi risente del limite di propagazione della luce. La comune pila ISO/OSI male si adatta a queste condizioni, soffrendo a più livelli delle peculiari caratteristiche della rete. In particolare la combinazione TCP/IP è resa inutilizzabile da questi impedimenti e questo argomento verrà approfondito nel prossimo capitolo. Varie soluzioni sono state proposte per affrontare questo problema, tra cui i Performance Enhancing Proxies (PEPs) che però, oltre a non rappresentare una vera soluzione, creano problemi di incompatibilità con gli attuali protocolli di sicurezza.
		
		Una valida soluzione alternativa è l'architettura DTN, che introduce un protocollo, chiamato Bundle Protocol (BP), sovrastante il livello di trasporto (TCP, UDP, etc..) o il livello fisico (Bluetooth, Ethernet, ...). Questa aggiunta permette la memorizzazione su lunghi periodi ai nodi intermedi, permettendo così di poter affrontare interruzioni del canale e lunghi ritardi. Dividendo il percorso end-to-end in più salti di tipo DTN si estende il concetto di TCP-splitting già utilizzato sui PEP, permettendo di usare protocolli specializzati necessari, ad esempio, sui link satellitari.
		
		\subsection{Limitazioni dei protocolli convenzionali in condizioni estreme}
		
		Per meglio comprendere le limitazioni che i protocolli standard di Internet hanno in condizioni di rete precarie verrà proposto un esempio. Consideriamo uno scenario composto da un centro di ricerca, un complesso di antenne e un satellite in orbita marziana. La situazione è schematizzata nella figura (ref to im). 
		
		Il percorso che divide il laboratorio dalle antenne per lo spazio profondo è una rete Internet convenzionale, caratterizzata da:
		\begin{itemize}
			\item Basse latenze, nell'ordine dei millisecondi.
			\item Alte velocità, fino a qualche Gb/s
			\item Comunicazione bidirezionale
			\item Connettività sorgente-destinazione continua
		\end{itemize}
		Quindi in questo tratto lo stack TCP/IP è usato nelle sue condizioni nominali, potendo offrendo tutti i servizi che conosciamo come frammentazione e ritrasmissione automatica. 
		
		Analizzando invece il secondo tratto di comunicazione, tra il complesso di antenne e l'orbiter marziano, ci accorgiamo che la situazione è ben diversa:
		\begin{itemize}
			\item Lunghe latenze di propagazione, nell'ordine di minuti
			\item Basse velocità, tipicamente qualche decina di Kb/s
			\item Interruzioni a causa di interferenze
			\item Connessione deterministicamente intermittente
		\end{itemize}
		Le lunghe latenze rendono inutilizzabile il meccanismo di ritrasmissione del TCP eccedendo abbondantemente i timeout propri del protocollo. Inoltre il meccanismo di handshake a tre stadi si protrarrebbe fino ad occupare gran parte dell'oportunità di contatto. Protocolli alternativi al TCP sono l'UDP, che però non offre meccanismi di ritrasmissione automatica, e altri più esotici che comunque non produrrebbero risultati soddisfacenti per le stesse ragioni del TCP.
		Risulta quindi che non è possibile usare un unico stack di protocolli su tutti i segmenti della rete, pur rimanendo nella necessità di avere un meccanismo di ritrasmissione automatico (ARQ). 
		Infine l'ultimo impedimento è la connessione intermittente che dai normali sistemi di routing è elaborata come una totale disconnessione eccedendo i tempi di timeout standard. Il nodo è quindi considerato strutturalmente perso piuttosto che in una disconnessione pianificata e di conseguenza un tipico calcolo di percorso basato su IP non è possibile.
		
		
		\subsection{Architettura DTN}		
		
		L'archittettura DTN è basata sull'introduzione di un nuovo strato protocollare a livello di trasporto o anche a livelli più bassi chiamato Bundle Protocol (BP). Il punto essenziale è dotare i nodi della capacità di gestire ritardi e disconnessioni, permettendo di memorizzare i dati localmente in attesa dell'opportunità per inoltrarli al prossimo nodo. 
		
		Il Bundle Protocol è in grado di interfacciarsi con i livelli inferiori, generalmente di trasporto, per mezzo di Convergence Layer Adapters (CLAs). Nel tempo sono stati definiti vari CLA, a cominciare dai protocolli di trasporto più comuni come TCP, UDP, LTP, passando poi anche nella seconda versione (DTN2) a protocolli di livello datalink come Bluetooth ed Ethernet. 
		In combinazione con il BP, ogni nodo può utilizzare il CLA più adatto per l'inoltro successivo. 
		
		L'uso del Bundle Protocol è guidato da alcuni principi di design a livello applicativo che contribuiscono ad ottenere migliori risultati:
		\begin{itemize}
			\item Gli applicativi sono tenuti a minimizzare gli scambi di andata e ritorno
			\item Gli applicativi dovrebbero essere in grado di gestire interruzioni improvvise mantenendo il trasferimento attivo
			\item Gli applicativi sono tenuti a specificare il tempo di validità dei dati e loro importanza relativa.
		\end{itemize}
		
		
		L'architettura DTN offre diverse nuove caratteristiche che saranno elencate e descritte di seguente.
		\begin{itemize}
			\item La DTN può agire come strato legante di tecnologie disomogenee, come reti wireless {\it ad hoc} di sensori, WLAN, link satellitari, internet, ecc. Affinché il BP sia correttamente gestito, sui nodi della rete è necessario che sia installato il Bundle Protocol Agent (BPA), che permetterà di dividere il percorso in vari hop DTN. Nei diversi hop potranno essere usati differenti CLA per adattarsi ai molteplici protocolli a cui una rete disomogenea può appoggiarsi. Oppure è anche possibile appoggiarsi a diverse varianti dello di un protocollo utilizzando lo stesso CLA in hop successivi. L'architettura multi-hop DTN può essere vista come una generalizzazione del TCP-splitting, aspetto che verrà approfondito nella prossima sottocapitolo.
			
			\item Possibilità di archiviazione ai nodi intermedi. Questa è un'importante differenza dallo stack TCP/IP tradizionale, dove, in presenza di reti convenzionali che assumono connettività persistente e brevi ritardi, i router intermedi trattengono l'informazione solo per brevi periodi lasciando la memorizzazione a lungo termine solo ai nodi finali. Questo è motivato dal fatto che appoggiandosi una connessione stabile si suppone che l'informazione possa essere recuperata direttamente dalla fonte. Tutto ciò ovviamente non è possibile nelle reti in condizioni estreme e, nell'ottica di riuscire ad affrontare i lunghi tempi di andata e ritorno come anche le interruzioni di canale, nelle reti DTN è necessario che i nodi intermedi abbiano la possibilità di memorizzare a lungo termine l'informazione. Questo contribuisce alla robustezza dell'architettura in presenza di disturbi, disconnessioni e problemi tecnici temporanei come ad esempio riavvi del nodo. Da un altro punto di vista però la memorizzazione a lungo termine può portare a congestionare i nodi, questa è una conseguenza da considerare e gestire.
			
			\item Tenendo a mente il contesto generale di una rete prona a non avere una connessione persistente tra sorgente e destinazione, l'architettura DTN stabilisce che il nodo avente i dati è responsabile per questi ultimi. Quindi eventuali ritrasmissioni che nelle reti convenzionali sono gestite dai nodi finali qui sono gestite da ogni singolo nodo intermedio.
			
			\item Late Binding: ogni nodo della rete DTN è identificato da un Endpoint Identifier (EID), che sintatticamente è rappresentato da un Uniform Resource Identifier (URI). Nel Bundle Protocol non esiste concetto di indirizzo, l'instradamento è basato puramente su EID. Quindi la risoluzione DNS di un nome destinazione può essere posposta (Late Binding) fino a che non si entra in contatto con la struttura necessaria e il percorso fatto finora è basato su EID. Questa è una caratteristica che permette di aggirare la mancanza di specifiche strutture come un DNS in reti isolate.
			
			\item Il routing in questa architettura deve coinvolgere priorità diverse dalle reti tradizionali, come ad esempio la capacità di memorizzazione e la gestione dell'energia. Certamente il tempo di consegna rimane un aspetto importante, ma in questo contesto può essere soggetto a compromessi. Infine il routing deve gestire il concetto di contatto, che si può definire come intervallo di tempo nel quale dei nodi possono scambiarsi dati e con una certa aspettativa di banda. La  quantificazione di tutti questi aspetti non è banale.
			
			\item L'architettura DTN definisce due tipi di frammentazione, proattiva e reattiva. La prima è adatta in situazioni dove la connessione è pianifica o deterministica (tempo di contatto) e si è a conoscenza della quantità di dati scambiabili per ogni finestra temporale (volume di contatto). Quindi quando la connessione è conosciuta a priori, come ad esempio in comunicazioni spaziali, è possibile frammentare grossi bundle pianificando con precisione i trasferimenti compatibilmente con il volume di contatto disponibile.
			Al contrario la frammentazione reattiva entra in gioco a posteriori, quando ad esempio la connessione viene interrotta frequentemente e sono necessarie ritrasmissioni.
			
			
		\end{itemize}
		
		
		\subsection{DTN come evoluzione del TCP splitting}
		L'accesso a internet tramite TCP/IP ha portato allo sviluppo dei PEP e in generale di {\it protocol boosters } per adattare questo stack in reti eterogenee. Il loro ruolo modifica attivamente il flusso sorgente-destinazione per adattare ai nodi TCP/IP i tratti con prestazioni povere, di fatto inducendoli a credere di avere a che fare con connessioni a prestazioni migliori. Gli esempi proposti saranno incentrati sulla comunicazione satellitare, essendo questa uno scenario di tipico utilizzo dei PEP. Il link satellitare infatti presenta forti latenze e altre peculiarità che rendono necessari protocolli specializzati come LTP. I PEP o alternativamente le DTN si occupano di spezzare la connessione a livello di trasporto (TCP) per permettere la sostituzione di quest'ultimo con LTP, ad esempio. 
		%non ho capito un accidente dei principi di fate sharing
		
		Principalmente sono possibili due configurazioni diverse di PEP: distribuiti ed integrati. I primi sono presenti da entrambi i lati del link satellitare, i secondi invece solo da un lato del link satellitare. 
		%Immagine presa dal vostro paper al riguardo
		La tipologia più comune di PEP effettua il TCP splitting, dividendo la connessione a livello di trasporto in due parti. Nei PEP distribuiti quindi si hanno tre connessioni diverse, dove la prima e l'ultima generalmente sono su TCP standard e rete cablata, mentre la seconda sul link satellitare ne usa una versione specializzata.
		Il corrispondente DTN è simile a quest'ultima configurazione, essendo che l'architettura prevede i nodi avere installata la gestione del BP.
		Nei PEP integrati invece la connessione è spezzata in sole due parti, la prima con su rete cablata con TCP convenzionale e la seconda con un protocollo specializzato che però non può essere radicalmente diverso ma compatibile con il TCP standard.
		%Immagine con i due differenti stack
		
		Riassumendo le principali similarità e differenze:
		\begin{itemize}
			\item Entrambi hanno due connessioni a livello di trasporto, una cablata e l'altra satellitare.
			\item Entrambi possono usare una variante del TCP specializzata per la connessione satellitare.
			\item La soluzione DTN richiede che i nodi abbiano installata la gestione del BP
			\item Il TCP splitting viola il principio di connessione end to end, perché i PEP agiscono sia al livello applicativo che di trasporto, facoltà in teoria riservata ai soli nodi finali. Nelle DTN questo problema è superato perché il ruolo del TCP è ridefinito essendo che ogni hop DTN è previsto sia una connessione a sé stante.
 		\end{itemize}
		
		
	
	\section{Nanosatellite DTN Network}
		
		\subsection{Framework}	

		L'idea generale su cui si basa questo lavoro è una costellazione di nanosatelliti operanti con il paradigma DTN. Il satellite diventa un contenitore di dati che con il suo movimento li trasporta fino a destinazione. In questo scenario sono possibili scambi di dati tra satelliti in orbita, ma non sono possibili contatti tra più di due satelliti. Il costo decisamente più contenuto dell'infrastruttura è la motivazione principale di una rete satellitare di questo genere, proposta come alternativa alle soluzioni commercialmente disponibili. Nei prossimi paragrafi verranno approfonditi aspetti tecnici come le diverse tipologie di satellite, orbita e infrastruttura. Infine sarà a approfondito l'uso di questa piattaforma per l'estensione della rete Intrnet a zone rurali.
		
		\subsection{Orbite e piattaforma satellitare}
		Per meglio comprendere la configurazione della nostra costellazione seguirà un breve approfondimento sulle differenti tipologie di orbita e satellite. Infine sarà esposta e motivata la nostra scelta.

			\subsubsection{Classificazione delle orbite}
			La meccanica orbitale di corpi naturali e artificiali è governata dalle tre leggi di Keplero. Comprendendo queste regole è possibile giustificare le caratteristiche dei differenti tipi di orbita usati per i satelliti artificiali. Le leggi sono valide nel caso la massa del corpo orbitante è trascurabile rispetto al corpo centrale e si possono trascurare le interazioni con corpi diversi oltre i due in esame. Queste approssimazioni sono valide nel nostro caso di studio e le leggi sono riformulate dal punto di vista di un satellite artificiale in orbita terrestre.
			\begin{enumerate}
				\item L'orbita di un satellite è un'ellisse con la Terra in uno dei due fuochi.
				\item Il segmento che unisce il centro della Terra con il satellite descrive aree uguali in tempi uguali.
				\item Il quadrato del tempo che il satellite impiega a percorrere l'orbita è proporzionale al cubo della distanza media dalla Terra.
			\end{enumerate}
			
			Intuitivamente la prima legge mostra come un'orbita circolare rappresenti un caso particolare di orbita e  come questo possa produrre comportamenti particolari. La seconda e terza invece legano la distanza dei due corpi a come il tempo di orbita è distribuito e a quanto ammonta, rispettivamente. Il concetto generale è quindi che più un satellite è distante più questo è lento nel procedere sulla sua orbita.
			
			La classificazione orbitale è basata sui seguenti parametri: altitudine, eccentricità, inclinazione, corpo centrale, sincronia. Sono elencati solo le classificazioni di interesse per i satelliti artificiali.
						
			{\large \bf ALTITUDINE}
			\begin{itemize}
				\item {\bf LEO - Low Earth Orbit}
				L'altitudine è compresa tra i 160km e 2000km, con orbite di circa 1-2 ore di durata. Ogni satellite compre solo una porzione della superficie terrestre quindi è necessaria una costellazione per assicurare una copertura globale. Per la relativa vicinanza alla superficie le comunicazioni sono a bassa latenza e non è necessaria grande potenza in trasmissione.
				\item {\bf MEO - Medium Earth Orbit}
				Altitudini comprese tra 5000km e 10000 km, tipicamente usate per i sistemi di posizionamento, osservazione della terra e più raramente telecomunicazioni. Il periodo orbitale è intorno alle 12 ore. Sono ancora richiesti più satelliti per una copertura globale.				
				\item {\bf GEO - Geostazionary Earh Orbit}
				Questo è un particolare tipo di orbita, posta a 35786 km sopra l'equatore, che permette di posizionare il satellite in un punto fisso rispetto alla superficie terrestre. La maggior parte dei satelliti per telecomunicazioni sfrutta questo tipo di orbita che però, data la grande distanza da terra, soffre di una lunga latenza.
			\end{itemize}
			
			{\large \bf ECCENTRICITÀ}	
			\begin{itemize}
				\item Orbita circolare, particolare caso dell'orbita ellittica, fanno parte di questo gruppo l'orbita GEO e l'orbita di trasferimento di Hoffman, usata per trasferire veicoli spaziali tra orbite diverse. In generale anche le orbite basse fanno parte di questa categoria, sempre con un certo grado di approssimazione.
				\item Orbite eccentriche, entrambe queste orbite sono usate per sistemi di comunicazione e militari, quasi esclusivamente da Russi. \begin{itemize}
					\item Molnya orbit
					\item Tundra orbit					
				\end{itemize}
				Importante notare come queste fruttino la seconda legge di Keplero per soffermarsi per più tempo possibile sopra una specifica zona terrestre. Infatti con una forte eccentricità si ha che il satellite trascorre la maggior parte del tempo nella zona più alta dell'orbita (Apogeo). 								
			\end{itemize}
			
			{\large \bf INCLINAZIONE}	
			L'inclinazione è l'angolo tra il piano equatoriale terreste e il piano orbitale del satellite.
			\begin{itemize}
				\item Orbita polare, con inclinazione prossima ai 90 gradi, passante quindi per i poli del pianeta.
				\item Polare sincrona solare, permette di passare sopra l'equatore sempre alla stessa ora locale. Utile per sistemi di immagine satellitari.
			\end{itemize}
			
			{\large \bf CORPO CENTRALE}	
			\begin{itemize}
				\item Geocentrica, orbitante attorno alla Terra.
				\item Eliocentrica, orbitante attorno al Sole
				\item Aerocentrica, orbitante attorno a Marte.
			\end{itemize}
		
		
			{\large \bf SINCRONIA}	
			\begin{itemize}
				\item Sincrona, avente periodo orbitale uguale al periodo di rotazione del corpo centrale. Le orbite GEO fanno parte di questo gruppo (GSO)
				\item Semi Sincrona, avente periodo di rotazione pari alla metà del periodo di rotazione del corpo centrale. Ad esempio sistemi di localizzazione in MEO.
			\end{itemize}
			
						
			\subsubsection{Satelliti artificiali}
			Un satellite per telecomunicazioni è di fatto una base ripetente che permette a una o più basi terrestri di scambiarsi informazione in varie forme. Una stazione di terra trasmette sul satellite alla frequenza di {\it UpLink}, questo riceve ed amplifica il segnale ritrasmettendolo alla frequenza di {\it DownLink} alle stazioni riceventi. 
			
			Quindi i sistemi di telecomunicazione satellitare sono composti essenzialmente da un'infrastruttura terrestre e il satellite in orbita. L'infrastruttura comprende anche il centro di controllo del satellite, che si occupa di tracciamento, telemetria e controllo. Al contrario di quanto si possa pensare, il tracciamento è uno dei compiti essenziali nella gestione del satellite. Oltre ad essere necessaria per confermare la posizione corretta del satellite dopo il lancio e per permettere ad altri di puntare le proprie antenne, il tracciamento rimane una priorità costante durante tutta la vita operativa della sonda. Perturbazioni orbitali tendono a scostare il satellite dalla posizione ideale, rendendo anche necessarie periodiche accensioni dei motori di manovra per riposizionarsi correttamente. Inoltre in orbite basse (LEO) è costantemente necessario compensare l'attrito atmosferico che tende a rallentare l'oggetto e di conseguenza contribuire al suo decadimento orbitale. 
			Mediamente un satellite commerciale in orbita alta ha un tempo di vita nell'ordine di una decina d'anni, mentre per uno in orbita bassa (LEO) questo tempo si dimezza. La principale limitazione costituita è dalla quantità di propellente a disposizione per le manovre di correzione orbitale, in orbita bassa queste sono molto più frequenti a causa del costante decadimento menzionato predentemente. Una volta che il satellite è o sta per essere inservibile, il suo destino dipende dall'orbita in cui è posizionato, in orbite basse decadrà nell'arco di pochi anni, disintegrandosi nel rientro atmosferico, in orbite alte invece tenderà a rimanere nel proprio assetto per l'eternità, non essendo presente nessuna forza che lo rallenti verso la superficie. Generalmente e quando possibile, il satellite viene spostato in un'orbita cimitero, in modo da non interferire con futuri lanci in quella particolare zona. 
			
			%%Elementi costituenti di un satellite
			
			Le comunicazioni satellitari hanno grande vantaggio su grandi distanze, essendo insensibili alla degradazione tipica di ponti radio ad alta frequenza a causa dell'atmosfera e alla dispersione che avviene su cavi.
			I principali {\bf vantaggi} sono sintetizzati a seguito. 
			\begin{itemize}
				\item Copertura globale, possibile cioè portare servizi di alta qualità in ogni angolo del pianeta tramite una costellazione satellitare. 
				\item Capacità, intesa come numero di comunicazioni contemporanee.
				\item Affidabilità, essendo costantemente attivo anche quando le infrastrutture terrestri collassano.
				\item Sicurezza, essendo già di per sé una rete privata, aggiungendo protocolli di sicurezza è possibile ottenere robustezza maggiore dei collegamenti terrestri.
				\item Scalabilità, perché aggiungere una nuova rete a quella esistente diventa immediato, anche in zone rurali, semplicemente allocando la banda richiesta e installando l'attrezzatura necessaria in loco.
				\item Dispiegamento rapido, al contrario di un'infrastruttura terrestre, il satellite può essere operativo in tempi relativamente rapidi.				
				\item Costi, pur essendo piuttosto elevato l'investimento iniziale della progettazione e costruzione di satelliti, questo si diluisce in un tempo lungo e permette di evitare costi ancora più proibitivi di costruzione e mantenimento di infrastrutture terrestri in ambienti ostili.				
			\end{itemize} 
		
			I principali {\bf svantaggi} invece sono costituiti dai vincoli dimensionali che impongono di conseguenza limitazioni sulla potenza di trasmissione e sui guadagni delle antenne.	Entrambi i fattori dipendono dai parametri orbitali e sull'investimento monetario. 
			\begin{itemize}
				\item Potenza. L'unica fonte di energia disponibile a bordo di un satellite sono generalmente i pannelli solari. Durante la guerra fredda era tipico avere anche satelliti ad energia nucleare con a bordo un generatore a radioisotopi, ma, oltre al costo elevato, era presente un alto rischio di contaminazione in caso di lancio fallimentare e nel rientro atmosferico a fine vita operativa.
				La bassa potenza disponibile a bordo è il principale collo di bottiglia per la potenza trasmissiva e quindi la qualità del segnale ricevuto sulla superficie. 
				\item Sensibilità in ricezione. Quest'ultima è limitata dalle dimensioni fisiche dell'apparato ricevente, che deve essere contenuto in ordine di rientrare nelle specifiche di lancio. A terra sono quindi necessarie strutture imponenti e grandi potenze per compensare questa mancanza. 
				\item Disponibilità. Fatta eccezione per i satelliti in orbita geosincrona, che rimangono il caso preferito in quanto sempre visibili da una determinata zona, il restante numero di satelliti pianifica le comunicazioni su base temporale per ogni zona d'interesse, specificando durata e qualità del contatto, essendo questi predicibili deterministicamente.
				 
			\end{itemize}
			
			\subsubsection{Alternative commerciali}
			Mentenere una copertura Internet globale non è semplice:
			\begin{itemize}
				\item Con una costellazione in GEO è possibile mentenere una connessione persistente con un puntamento fisso, inoltre la costellazione può essere composta da un numero minore di satelliti rispetto a quelli necessari in orbita LEO, con l'altitudine maggiore come motivazione diretta di questo fatto.
				D'altra parte altezze più elevate comportano costi di lancio e piattaforma satellitare maggiori, infatti un satellite in questa orbita generelamente richiede un ivestimento intorno ai 300 milioni di dollari. Infine le zone polari non sono coperte per la posizione equatoriale dell'orbita GEO e le latenze sono alte a causa delle grandi distanze in gioco.
				
				\item Le costellazioni LEO sono in grado di mantenere una coperura globale sostituendosi a vincenda nella copertura durante il loro costante movimento, rendendo necessarie però antenne a puntamento automatico per ottenere velocità di trasferimento comparabili a quelle dei satelliti in GEO. Immaginando l'antenna del satellite puntata in direzione del suolo, ci si rende conto come la massima qualità di ricezione, ottenibile quando le due antenne, suolo e satellite, sono allineate, possa durare al massimo qualche istante. Lo svantaggio della minor banda viene compensato da una latenza decisamente minore. L'investimento si abbassa anch'esso, ma rimane comunque alto intorno ai 150 - 200 milioni di dollari, tenendo a mente che sono necessari più satelliti.
			\end{itemize}
			
			L'alternativa a queste due opzioni è rappresentata dai CubeSat, un tipo standardizzato di nanosatellite, in termini di volume e peso. Nello specifico, l'unità base del CubeSat è un parallelepipedo 10x10x11.35 cm, con un peso massimo di 1.33Kg, pensata per mettere a disposizione un litro di volume all'interno. Le unità base possono essere combinate per ottenere CubeSat più capienti. La standardizzazione, come per altro in ogni ambito ingegneristico, permette di abbattere ulteriormente i costi di lancio, non essendo necessario sviluppare una piattaforma nuova per ogni nanosatellite e semplificare le interazioni tra i gruppi che si occupano del lancio e della costruzione del nanosatellite.
			
			Il principale vantaggio nell'uso dei nanosatelliti è il costo di investimento decisamente abbattuto. Il costo di assembleaggio stimato si aggira intorno ai 50000\$ - 100000\$, mentre il costo di lancio stimato di tre CubeSat è all'incirca di 200000\$. Il costo totale, approssimativo, di una rete composta da 150 nanosatelliti e 3000 stazioni di terra si può stimare essere all'incirca di 33 milioni di dollari con un'aspettativa operativa di 5 anni. A 24 Mbps il volume totale scambiato sarebbe di 225 TB, per un costo al MB di 0.13\$. 
			%%Calcolo che non comprendo appieno onestamente
			Per confronto, vengono riportati alcuni esempi di tariffe commerciali (2009):
			\begin{itemize}
				\item {\it ORBCOMM}, servizio tipicamente utilizzato per brevi messaggi, propone 1000 caratteri a 1.40\$, tradotto in nostri termini 1433.60\$ per MB.
				\item {\it Iridium} offre un piano di connessione a tempo, 5000 minuti, utilizzabili in due anni, per 4994.99\$, circa 1\$ al minuto. Essendo che la velocità offerta era all'incirca 2400bps, traducibile in 1MB all'ora, equivale a 60\$ per MB.
				\item {\it INMARSAT}, che usa una costellazione di tre satelliti in GEO, offre una tariffa al MB di 7.50\$
			\end{itemize}
			%%Verificare un secondo questi dati
			
			\subsubsection{La nostra costellazione}
			La costellazione di satelliti scelta per questo lavoro è costituita da satelliti in orbita LEO, che a confronto con un'orbita GEO offre il vantaggio di investimenti iniziali e latenze minori, ma ha il principale svantaggio che i satelliti sono in costante movimento rispetto alla superficie e quindi l'apparato di terra deve essere in grado di orientare l'antenna per rimanere agganciato al segnale. Questa configurazione non permette, in termini di qualità del segnale, prestazioni ottimali in generale o comunque per tutta la durata del contatto. Dal punto di vista del satellite invece il puntamento è ad una generica, ma nota, inclinazione verso terra. Rimane vero però che un singolo satellite in orbita GEO costituisce un punto di criticità in caso di malfunzionamento rispetto ad una costellazione sparsa in orbita LEO. Questi compromessi però sono tutti a vantaggio del requisito principale di questo lavoro, ridurre i costi d'investimento per la creazione di una rete DTN in grado di estendere la copertura Internet a zone rurali. Una costellazione in orbita GEO rimane preferita in ambito di telecomunicazioni, ma comporta costi estremamente elevati. Sempre in quest'ottica è stata fatta la scelta del tipo di satellite, il nanosatellite.
			
			I nanosatelliti sono una realtà piuttosto recente, spesso utilizzati come piattaforme di test per nuove tecnologie o come piattaforma orbitale congiunta di università per esperimenti o dimostrazioni. Le loro peculiarità sono le dimensioni estremamente ridotte e standardizzate, nell'ordine delle decine di centimetri per lato, e il basso costo di progettazione, produzione e mantenimento. Eliminando completamente caratteristiche come un propulsore, pannelli solari orientabili, antenne ad alto guadagno, sistemi ridondanti, ecc... si possono ridurre notevolmente dimensioni e costi. In particolare i costi di messa in orbita sono particolarmente abbattuti in quanto possono essere aggregati ad altri lanci come carichi secondari. Infine alcuni sono lanciati insieme al rifornimento periodico alla Stazione Spaziale Internazionale (ISS) e da essa catapultati in orbita tramite un apposito lanciatore. 
			
			%%Chiedere a fabio maggiori informazioni sui parametri orbitali 
			%%Inserire i confronti con le attuali proposte commerciali e la motivazione di scelta di questi parametri orbitali.
			
			
		
		
		\subsection{Architettura di rete}	
		
		La rete proposta è un complesso costituito da un gruppo eterogeneo di nodi DTN, ciascuno avente un preciso ruolo e posizione nello scambio di dati. Questa rete è un'evoluzione diretta di quella proposta in (Inserire reference al papiro). Lo scheletro è composto da una costellazione di nanosatelliti con caratteristiche descritte nel capitolo precedente, corredato da due tipologie di stazioni terrestri adibite alla comunicazione satellitare. 
		
		Lo scenario di riferimento è basato sulle seguenti tipologie di nodi, disposti secondo l'ordine di invio verso gli utenti finali:
		\begin{itemize}
			\item {\bf Central Node, CN}
			
			Questo nodo svolge la funzione di controllore della rete, gestendo le richieste e i instradando flussi di traffico. Si occupa inoltre dell'interfaccia tra la rete DTN e la rete Internet convenzionale.
			
			\item {\bf HotSpots, HS}
			
			Gli HotSpots sono le basi di comunicazione satellitare connesse al Central Node. Dispongono quindi di un collegamento radio adibito alla comunicazione con i nanosatelliti e di un collegamento cablato con il CN, quest'ultimo potrebbe essere attuato anche attraverso la rete Internet convenzionale. 
			
			\item {\bf NanoSatellites, NS}
			
			I nanosatelliti attuano la parte di trasporto vera e propria dell'informazione, assumendo il ruolo di muletto dei dati caricando e scaricando dati alle stazioni di terra.
			
			\item {\bf ColdSpots, CS}
			
			I CS sono l'analogo degli HS per le zone rurali. Sono localizzati in zone strategiche nelle prossimità degli utenti finali, e sono collegati con essi tramite una rete locale su cui non ci sono particolari vincoli. Infine è presente l'interfaccia radio per la comunicazione satellitare.
			
			\item {\bf Rural Nodes, RN}
			
			Sono i veri e propri utenti di questa rete. I nodi rurali sono un generici dispositivi a cui siamo avvezzi nella nostra quotidianità, come PC, smarthphone e tablet. Questi nodi creano le richieste e ricevono le risposte, sono quindi i nodi finali.  
						
		\end{itemize}
		
		Ora che sono stati esposti i vari costituenti della rete, è possibile esporre come una tipica comunicazione avviene in questo contesto, focalizzandosi sugli aspetti relativi alla natura DTN. 
		
		I nodi rurali sono composti da normali utenti che sono interessati ad usufruire di servizi base come la consultazione della propria posta o la navigazione di pagine web. La loro interfaccia con la rete Internet è costituita dal ColdSpot locale, ad esso le richieste vengono inoltrate e da qui in poi entra in gioco il paradigma DTN. Non esistendo una connessione diretta e stabile con la destinazione, rappresentata ad esempio da un server adibito alla pagina web richiesta, il CS trattiene la richiesta dell'utente ed elabora una risposta che emula il comportamento del server a cui l'utente rurale dovrebbe essere connesso, in modo da soddisfare i criteri di timeout tipici di una rete Internet. In questo modo il terminale dell'utente è portato a comportarsi normalmente come se fosse realmente connesso alla destinazione richiesta. Intanto il CS incapsula i dati da inviare in un Bundle DTN e calcola il miglior modo per recapitare i dati ad un HS. Dettagli riguardo a come questa scelta è elaborata saranno investigati nel prossimo capitolo. Quindi la richiesta è trattenuta fino all'opportunità di contatto con il nanosatellite prescelto. Il dato viene caricato a bordo del satellite e qui nuovamente conservato fino ad essere scaricato sul nodo HS prescelto. Ricordiamo che i nodi HS sono le stazioni radio connesse al nodo centrale tramite rete veloce. Da qui la richiesta è immediatamente inoltrata al nodo centrale, che si occupa di recuperare i dati dell'ipotetico sito come fosse un generico utente di Internet. Segue quindi l'iter standard di un nodo Internet alle prese con il recuperare una pagina web, interrogando prima il DNS, instanziando tutte le connessioni del caso, infine recuperando i dati. A questo punto è tempo di fare la strada inversa verso il nodo rurale, nuovamente, le scelte di instradamento del dato saranno oggetto del prossimo capitolo. I dati recuperati verranno mandati ad un HotSot, il quale analogamente al ColdSpot in fase di richiesta, tratterà il tutto fino a quando il NanoSatellite non entrerà in contatto. Il pacchetto DTN a questo punto sarà caricato sul NS e trasportato lungo l'orbita, qui potrà saltare per un numero indefinito di volte tra NS che entrano in contatto tra di loro, sempre seguendo una logica di instradamento ben precisa. L'ultimo salto sarà verso terra al CS in ricezione della zona rurale di partenza. Qui il CS si occuperà di instradare i dati al nodo di partenza, finalizzando la connessione.
		%%inserire grafico di comunicazione
		Da questo esempio di comunicazione bidirezionale si può dedurre che i nodi applicanti il paradigma DTN sono gli HotSpots, i Nanosatelliti e i ColdSpots. Questi spezzano le connessioni a livello di trasporto nella logica di evoluzione del TCP splitting esposta nel capitolo 2.4 e attuano la memorizzazione dei dati esposta nel capitolo 2.3. Il Central Node è un nodo capace di lavorare al livello del Bundle Protocol, gli altri nodi della rete, come ad esempio i nodi Internet che collegano gli HS al CN, ignorano il BP assimilandolo ad un livello applicativo. 
	
		
		\subsection{Connessioni alla Rete in aree rurali}
			
			L'accessibilità della conoscenza è diventata una delle pietre miliari del nostro tempo, uno dei simboli più marcati del progresso portato dall'evoluzione impetuosa del settore delle Comunicazioni e delle Tecnologie dell'Informazione (ICT). Questo progresso però si è affermato principalmente nelle aree più sviluppate, andando inoltre a migliorare enormemente le infrastrutture burocratiche e organizzative. Il grande freno nell'affermazione di tali tecnologie in zone più disagiate o rurali è il costo dell'infrastruttura di supporto, che richiederebbe creare da zero una rete in zone dalle condizioni naturali estremamente avverse o dalla situazione politica incerta e disinteressata. Inoltre la scarsa densità abitativa di tali aree non potrebbe giustificare a maggior ragione il dispiegamento di una rete ad alta capacità. Esclusi luoghi dalle condizioni naturali estreme, come latitudini polari o atolli oceanici, le restanti zone rurali potrebbero essere caratterizzate inoltre da problemi di emergenza umanitaria ben più gravi della mancanza dell'infrastruttura ICT, ma se posta in questi termini il divario con la civiltà urbana occidentalizzata non potrebbe che aumentare. Nel prossimo capitolo saranno esposti alcuni approcci sviluppati per creare un'alternativa alla rete convenzionale che noi conosciamo nella nostra quotidianità.
			\subsubsection{Altri approcci al problema}
			
			\begin{itemize}
				\item {\bf Digital Gangetic Plains}
				
				Questo progetto consiste nel coprire una vasta area di territorio con una rete WiFi 802.11. I vantaggi di questa tecnologia è che non ha costi proibitivi, occupa una porzione libera nello spettro di frequenze e offre alte velocità di connessione. La rete è composta da una serie di ripetitori WiFi atti a creare una rete {\it mesh}, cioè una rete connessa a grafo. Il nodo principale ha accesso alla rete Internet cablata e rappresenta il punto d'accesso. I ponti radio possono essere distanti decine di kilometri e sono collegati da antenne direzionali ad alto guadagno. 
				
				\item {\bf DakNet}
				
				Questa soluzione coinvolge l'utilizzo di infrastrutture esistenti per il trasporto fisico dei dati, nello specifico mezzi pubblici come autobus. Il veicolo diventa la spola, similmente al nanosatellite del nostro caso, tra la zona rurale e il punto di accesso a Internet. Muovendosi lungo il suo percorso programmato, il mezzo prende in carico le richieste da parte degli utenti sparsi per i villaggi e una volta arrivato in contatto con il punto di accesso recupera tutti i dati per consegnarli al prossimo giro. Il cuore pulsante a bordo del bus è un PC Linux embedded e comunicazione WiFi. 
				
				\item {\bf KioskNet}
				
				Il termine {\it kiosk} è traducibile come "chiosco" ma prende l'accezione specifica di una sorta di internet caffè, utilizzabile a pagamento per l'accesso a Internet pubblico. Questo genere di attività è fiorente in zone rurali, ma soffre di problemi legati all'infrastruttura fatiscente che lo supporta, cominciando dall'energia elettrica per finire alla connessione telematica. Lo scopo quindi è stato di semplificare la tecnologia supportante questo servizio per contenere costi e manutenzione, cominciando con l'usare PC embedded a basso consumo per potersi rendere indipendenti con l'utilizzo di pannelli solari, continuando poi adottando un paradigma DTN con idea simile al DakNet in quanto a riutilizzo dei sistemi di trasporto esistenti.
								
			\end{itemize}
			
		Il limite primo di tutte queste proposte è l'affrontare il problema con un approccio locale, sempre a rischio di fattori ambientali e politici instabili. L'alternativa proposta da questo lavoro di ricerca è chiamata "Ring Road" e non è altro che l'infrastruttura esposta ad inizio capitolo.
			
		\subsubsection{Motivazione dell'approccio nanosatellitare}			

		Questa ricerca punta alla creazione di una rete globale e indipendente, i cui principali vantaggi sono in primo luogo il basso costo, sempre in paragone con le alternative esistenti che propongono servizi comparabili esposti precedentemente, la scalabilità, essendo che è possibile un dispiegamento graduale delle risorse, aggiungendo man mano che i fondi diventano disponibili nuovi nanosatelliti, HotSpots e zone rurali. Da questo punto si ottiene un'importante altra conseguenza, la robustezza della rete. La perdita di uno o più nodi non costituisce una criticità per il funzionamento complessivo. 
			
		Altri vantaggi nell'usare una rete satellitare derivano dalla totale indipendenza dalle condizioni e vicissitudini superficiali, quindi la rete è immune a disastri naturali e difficilmente soggetta a sabotaggi e censure politiche.
			
		
		
	\section{Source Contact Graph Routing}
	
		\subsection{Introduzione}
		Per instradamento, in inglese {\it routing}, si intende il percorso con cui i dati vengono trasferiti tra sorgente e destinazione. Relativamente a questo lavoro la questione è stata rimandata ad un capitolo separato perché rappresenta il punto centrale della tesi, posta nel contesto di ricerca più ampio introdotto in precedenza. 
		
		Verranno in seguito esposte le nozioni necessarie per comprendere il concetto Source Contact Graph Routing (SCGR), traducibile come instradamento a sorgente su grafo di contatti. SCGR è un'estensione del concetto di CGR proposto da Burleigh et al. \cite{burleigh2010contact}, che trae vantaggio dal fatto che le missioni spaziali sono minuziosamente pianificate e quindi i vari nodi possono essere a conoscenza di un futuro stato della rete. Non si rende necessario un dialogo tra nodi prima di confermare il prossimo salto, ma piuttosto la conoscenza dello stato della rete permette di creare un grafo di contatti, inteso come un modello variante nel tempo di connettività tra nodi, sul quale è possibile applicare l'algoritmo euristico proposto per trovare il miglior percorso. Con algoritmo euristico si intende essere capace di trovare una soluzione prossima all'ottimale, un compromesso necessario per limitare i tempi di calcolo necessari per una soluzione ottima. Sviluppata dalla NASA, questa idea permette maggiore flessibilità e resilienza della rete spaziale, permettendo di semplificare e ridurre i costi di gestione della missione.
		
		L'iniziale soluzione applicata alla rete DTN nanosatellitare per l'instradamento seguiva la logica CGR, dove ogni nodo conteneva le informazioni di contatto della rete e decideva ogni hop successivo durante le opportunità di contatto. L'alternativa proposta da questo lavoro invece sposta la decisione dell'intero percorso del bundle nel momento in cui questo viene generato. Il principale vantaggio consiste che all'origine si può ancora avere una visione d'insieme completa, non essendo condizionati nelle scelte di instradamento precedenti. Da qui l'aggiunta di "Source" a CGR, essendo un instradamento deciso a livello di sorgente. 
		
		\subsection{Predizione dei contatti}
		%Spiegazione contact table e la sua logica di compilazione
		
		
		\subsection{Concetti di una rete deterministica}
		%%vantaggi di una rete deterministica in termini di gestione e ottimizzazione
		\subsection{Algoritmo di instradamento}
		
	\section{Risultati simulazione e analisi delle prestazioni}
		
		\subsection{Simulatore}
		%ns3 descrizione

		\subsection{Simulazione della rete}
		%i compromessi e adattamenti della rete al simulatore
			
		\subsection{Scenari di simulazione}
		%i diversi scenari spiegati e motivati
			
		\subsection{Risultati}
		%tabelle e grafici con i risultati
	
	\section{Conclusione e sviluppi futuri}
	%riassunto dei risultati e idee future di sviluppo
	
	\bibliographystyle{ieeetr}
	\bibliography{references}
	
\end{document}